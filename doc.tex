\documentclass[12pt,a4paper]{book}
\usepackage[minitoc]{teach}
\usepackage[utf8]{inputenc}
\usepackage[french]{babel}
\usepackage[T1]{fontenc}
\usepackage{amsmath}
\usepackage{amsfonts}
\usepackage{amssymb}
\usepackage{graphicx}
\renewcommand{\headrulewidth}{0pt}
\renewcommand{\footrulewidth}{0pt}
\fancyfoot[C]{\thepage}


\author{YAWO Kossi Atsu}
\newcommand{\prof}{YAWO Kossi Atsu}
\newcommand{\matiere}{MATHEMATIQUES}
\newcommand{\classe}{3$^{ème}$}
\title{Mes devoirs de Mathématiques}
\begin{document}
\begin{td}{\matiere}{\classe}{24 mai 2019}{\prof}
\end{td}

\newpage
\textbf{\underline{CONTROLE DE \matiere}}  \qquad \qquad  \qquad \qquad \emph{Durée: 1h} \qquad \qquad \emph{Noté sur: 10} \\
\par
\underline{Exercice 1} \emph{(4pts)}\\
On a relevé les âges des élèves d'une classe de 3ème:\\
13 \quad 16 \quad 14 \quad 15 \quad 13 \quad 15 \quad 16 \quad 14 \quad 14 \quad 15 \quad 16 \quad 14 \quad 15 \quad 15 \quad 14 \quad 17 \quad 14 \quad 15 \quad 13 \quad 14 \quad 15 \quad 16 \quad 15 \quad 16 \quad 14 \quad 15 \quad 13 \quad 14 \quad 15 \quad 17 \quad 16 \quad 15 \quad 16 \quad 15 \quad 15 \quad 14 \quad 15 \quad 13 \quad 14 \quad 16.
\begin{enumerate}
\item Établissez le tableau des effectifs des âges de ces élèves. \emph{(1,5pt)}
\item Quel est le mode de cette série statistique? \emph{(0,5pt)}
\item Construisez le diagramme en bâton de cette série. \emph{(1pt)}
\item Calculez l'âge moyen des élèves de cette série. \emph{(1pt)}
\end{enumerate}
\vspace{0,5cm}
\underline{Exercice 2} \emph{(6pts)} \\
Dans le plan muni d'un repère orthonormé (O, I, J), on considère les points A, B et C tel que: $\overrightarrow{OA}=7\overrightarrow{OI}+\overrightarrow{OJ}$ ; $\overrightarrow{OB}=8\overrightarrow{OI}+4\overrightarrow{OJ}$ et $\overrightarrow{CO}=\overrightarrow{OI}-7\overrightarrow{OJ}$.
\begin{enumerate}
\item Place les points A, B et C dans le repère. \emph{(0,5pt)}
\item \begin{enumerate}
\item Montre que les vecteurs $\overrightarrow{AB}$ et $\overrightarrow{BC}$ sont orthogonaux. \emph{(1pt)}.
\item Donne en justifie la nature du triangle ABC. \emph{(0,5pt)}
\end{enumerate}
\item Soient le point M milieu du segment [AC] et le point D symétrique de B par rapport à M.
\begin{enumerate}
\item Détermine les coordonnées de M et de D. \emph{(1pt)}
\item Précise la nature du quadrilatère ABCD. Justifie. \emph{(1pt)}
\end{enumerate} 
\item \begin{enumerate}
\item Construis le cercle $\mathcal{C}$ circonscrit au quadrilatère ABCD. \emph{(0,5pt)}
\item Précise son centre, calcule son rayon et montre qu'il passe par le point O. \emph{(1,5pt)}
\end{enumerate}
\end{enumerate}
\vspace{0,5cm}
\hrule
\vspace{0,5cm}

\textbf{\underline{CONTROLE DE \matiere}} \qquad \qquad  \qquad \qquad \emph{Durée: 1h} \qquad \qquad \emph{Noté sur: 10} \\
\par
\underline{Exercice 1} \emph{(4pts)}\\
On a relevé les âges des élèves d'une classe de 3ème:\\
13 \quad 16 \quad 14 \quad 15 \quad 13 \quad 15 \quad 16 \quad 14 \quad 14 \quad 15 \quad 16 \quad 14 \quad 15 \quad 15 \quad 14 \quad 17 \quad 14 \quad 15 \quad 13 \quad 14 \quad 15 \quad 16 \quad 15 \quad 16 \quad 14 \quad 15 \quad 13 \quad 14 \quad 15 \quad 17 \quad 16 \quad 15 \quad 16 \quad 15 \quad 15 \quad 14 \quad 15 \quad 13 \quad 14 \quad 16.
\begin{enumerate}
\item Établissez le tableau des effectifs des âges de ces élèves. \emph{(1,5pt)}
\item Quel est le mode de cette série statistique? \emph{(0,5pt)}
\item Construisez le diagramme en bâton de cette série. \emph{(1pt)}
\item Calculez l'âge moyen des élèves de cette série. \emph{(1pt)}
\end{enumerate}
\vspace{0,5cm}
\underline{Exercice 2} \emph{(6pts)} \\
Dans le plan muni d'un repère orthonormé (O, I, J), on considère les points A, B et C tel que: $\overrightarrow{OA}=7\overrightarrow{OI}+\overrightarrow{OJ}$ ; $\overrightarrow{OB}=8\overrightarrow{OI}+4\overrightarrow{OJ}$ et $\overrightarrow{CO}=\overrightarrow{OI}-7\overrightarrow{OJ}$.
\begin{enumerate}
\item Place les points A, B et C dans le repère. \emph{(0,5pt)}
\item \begin{enumerate}
\item Montre que les vecteurs $\overrightarrow{AB}$ et $\overrightarrow{BC}$ sont orthogonaux. \emph{(1pt)}.
\item Donne en justifie la nature du triangle ABC. \emph{(0,5pt)}
\end{enumerate}
\item Soient le point M milieu du segment [AC] et le point D symétrique de B par rapport à M.
\begin{enumerate}
\item Détermine les coordonnées de M et de D. \emph{(1pt)}
\item Précise la nature du quadrilatère ABCD. Justifie. \emph{(1pt)}
\end{enumerate} 
\item \begin{enumerate}
\item Construis le cercle $\mathcal{C}$ circonscrit au quadrilatère ABCD. \emph{(0,5pt)}
\item Précise son centre, calcule son rayon et montre qu'il passe par le point O. \emph{(1,5pt)}
\end{enumerate}
\end{enumerate}

\end{document}
